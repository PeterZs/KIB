% Template for NIME 2013
%
% Modified by Kyogu Lee on 7 October 2012
%
% Modified by Georg Essl on 7 November 2011
%
% Based on "sig-alternate.tex" V1.9 April 2009
% This file should be compiled with "nime2011.cls"
%

\documentclass{nime-alternate}

\begin{document}
%
% --- Author Metadata here ---
\conferenceinfo{NIME'13,}{May 27 -- 30, 2013, KAIST, Daejeon, Korea.}

\title{KIB: Simplifying Gestural Instrument Creation using Widgets}

%
% You need the command \numberofauthors to handle the 'placement
% and alignment' of the authors beneath the title.
%
% For aesthetic reasons, we recommend 'three authors at a time'
% i.e. three 'name/affiliation blocks' be placed beneath the title.
%
% NOTE: You are NOT restricted in how many 'rows' of
% "name/affiliations" may appear. We just ask that you restrict
% the number of 'columns' to three.
%
% Because of the available 'opening page real-estate'
% we ask you to refrain from putting more than six authors
% (two rows with three columns) beneath the article title.
% More than six makes the first-page appear very cluttered indeed.
%
% Use the \alignauthor commands to handle the names
% and affiliations for an 'aesthetic maximum' of six authors.
% Add names, affiliations, addresses for
% the seventh etc. author(s) as the argument for the
% \additionalauthors command.
% These 'additional authors' will be output/set for you
% without further effort on your part as the last section in
% the body of your article BEFORE References or any Appendices.

\numberofauthors{1} %  in this sample file, there are a *total*
% of EIGHT authors. SIX appear on the 'first-page' (for formatting
% reasons) and the remaining two appear in the \additionalauthors section.
%
\author{
% You can go ahead and credit any number of authors here,
% e.g. one 'row of three' or two rows (consisting of one row of three
% and a second row of one, two or three).
%
% The command \alignauthor (no curly braces needed) should
% precede each author name, affiliation/snail-mail address and
% e-mail address. Additionally, tag each line of
% affiliation/address with \affaddr, and tag the
% e-mail address with \email.
%
% 1st. author
\alignauthor Edward Zhang\\
       \affaddr{Princeton University}\\
       \affaddr{Princeton, NJ}\\
       \email{edwardz@princeton.edu}
}
\maketitle
\begin{abstract}
% FIXME: Abstract
The Microsoft Kinect is a popular and versatile input device for creating musical interfaces.
However, using the Kinect for such interfaces requires not only significant programming
experience, but also complex geometry or machine learning techniques to translate joint
positions into higher level gestures. We designed the Kinect Instrument Builder (KIB) to
address these difficulties by structuring gestural interfaces as combinations of gestural
widgets. KIB allows the user to design an instrument by combining simple gestural primitives,
each with a set of simple but exciting visual feedback elements. After designing an instrument,
KIB allows users to play the instrument, displaying visualizations and transmitting OSC 
messages for synthesis or further mapping.
\end{abstract}

\keywords{Kinect, gesture, widgets, OSC, mapping}

\section{Introduction}
The Microsoft Kinect is a popular and versatile input device for creating musical interfaces.
However, using the Kinect for such interfaces requires not only significant programming
experience, but also complex geometry or machine learning techniques to translate joint
positions into higher level gestures. We designed the Kinect Instrument Builder (KIB) to
address these difficulties by structuring gestural interfaces as combinations of gestural
widgets. KIB allows the user to design an instrument by combining simple gestural primitives,
each with a set of simple but exciting visual feedback elements. After designing an instrument,
KIB allows users to play the instrument, displaying visualizations and transmitting OSC 
messages for synthesis or further mapping.
%FIXME: Introduction
\section{Background}
\subsection{Widget-based Interface Design}
\subsection{Gestural Widgets}
%FIXME: Backgroud
\section{Implementation}
%FIXME: Implementation
KIB consists of two distinct applications. The instrument design interface is a web
application that allows the user to rapidly construct a widget-based instrument. The user
can save the resulting instrument and open it using the performance interface.
\subsection{Instrument Design Interface}
\subsection{Performance Interface}
\section{Evaluation}
%FIXME: Evaluation
\section{Discussion}
%FIXME: Discussion
\section{Conclusion}
%FIXME: Conclusion
\end{document}  % This is where a 'short' article might terminate

% The following two commands are all you need in the
% initial runs of your .tex file to
% produce the bibliography for the citations in your paper.
\bibliographystyle{abbrv}
\bibliography{kib-references}  % sigproc.bib is the name of the Bibliography in this case
% You must have a proper ".bib" file
%  and remember to run:
% latex bibtex latex latex
% to resolve all references
%
% ACM needs 'a single self-contained file'!
